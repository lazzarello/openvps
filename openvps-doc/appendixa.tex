\chapter{Details of the OpenVPS host scripts\label{app-openvps-host}}

\begin{itemize}

\item
  Configuration of the physical server to make it suitable for hosting.

  \begin{itemize}

  \item
    Interactively sreate an editable OpenVPS configuration file which
    contains certain defaults for virtual servers (backups, SSL cert
    defaults, location of vserver utils, etc.).
    
  \item
    Up the amount of available Sys V IPC semaphores, as the defaults
    provided in Linux are too low for virtual hosting.

  \item
    Disable IPv6 support. (This may be a temporary measure until the
    Linux VServer IPv6 is figured out).

  \item
    Disable all of the services on the machine except for the absolute
    minimal set required.

  \item
    Configure sshd and httpd to only listen on the main server IP rather
    than 0.0.0.0 (default).

  \item
    Configure a caching name server which the virtual servers on this
    physical server will be using. Make sure the nameserver only
    listens on the main IP.

  \item
    Configure the util-vserver vprocunhide service to run at startup.

  \item
    Add a user called ``ohd''. This user executes certain helper
    programs that must run outside of virtual servers because they
    require elvated privileges (e.g. traceroute).

  \item
    Add a user called ``backup''. This user is used to accept backups
    from other servers when they are configured in a peer
    configuration.

  \end{itemize}

  \item
    Building of a \dfn{Reference Server}. The reference server serves
    as the starting template for all virtual servers. The reference
    server consists of a large selection of Fedora packages chosen
    based on applicability to a hosting environment. This list can be
    augmented via the OpenVPS configuration file.

    \begin{itemize}
      
    \item
      Make the necessary limited set of device files.

    \item
      Resolve the initial list of packages - the latest
      version/release will be chosen from the available list, which
      can be a local directory or a URL of a Fedora mirror.

    \item
      Install the specified packages, including some if the
      customized RPM's for the OpenVPS (e.g. webmin with all
      physical-server-specific options removed).

    \item
      Turn off all services except those that are most appropriate
      in a VPS.

    \item
      Make vserver-friendly fstab and mtab files.

    \item
      Fix up the /etc/init.d/halt script to behave error-free in the
      vserver environment.

    \item
      Remove klogd references from the syslogs service.
      
    \item
      Create a /usr/libexec/oh with some OpenVPS-specific binaries
      (traceroute).

    \item
      Fix up the /etc/sysconfig/i18n file.
	
    \item
      Remove mingetty's from /etc/inittab.
      
    \item
      Fix up vncserver to use the light-weight xfce desktop.
      
    \item
      Run the ``fixflag'' OpenVPS command to set appropriate
      immutable/iunlink flags for non-configuration files and as
      described in OpenVPS clone rules.
      
    \end{itemize}

  \item
    \dfn{Cloning} of the Reference Server. This is a process by which
    the virtual server is created. The actual cloning means either
    copying, hard-linking or touching a file depending on its VServer
    flags and the OpenVPS clone rules. (The iunlink flag is part of
    the VServer \dfn{unification} feature).

  \item
    \dfn{Customization} of the newly-cloned VPS. This is where the
    final touches are applied.

    \begin{itemize}
      
    \item
      Create the VServer config for the VPS. This is where the IP,
      the root directory, the scheduling limits and other things
      are set up.
      
    \item
	Add a user account. This account typically matches the name of
	the VPS and is known as the \dfn{main account}. 

    \item
      Set the initial password for the main account and root. The
      password can be supplied in plain text or as an MD5 hash.

    \item
      Make the /etc/resolv.conf file which uses 127.0.0.1 (the
      physical server) as the primary DNS server and searches a
      supplied domain.

    \item
      Write an /etc/motd file (configurable).

    \item
      Fix-up the /etc/rc.d/rc to return true to play nice with the
      VServer utils.

    \item
      Configure sendmail to accept mail for a supplied domain. Also
      up the RefuseLA setting to prevent sendmail from stopping
      accepting mail at high server loads.

    \item
      Configure an IMAP and POP servers (currently dovecot) to only
      accept SSL connections.

    \item
      Create a stub index.html page (configurable).

    \item
      Set a disk limit for the VPS using VServer dlimimt tool.
      
    \item
      Create an RRD database and iptable rules for bandwidth
      accounting.

    \item
      Generate SSL certificates for the VPS (CA configurable).

    \item
      Create an Apache proxy URL for access to the admin panel.

    \item
      Create a CVS root (required by webmin).

    \item
      Disable PAM limits (work-around what seems to be a PAM bug).

    \item
      Set the webmin password to the same one as the main user.
	
    \item
      Generate an SSH key that will allow connecting to the ohd
      account to execute elevated priv programs (traceroute).
	
    \item
      Make lib/modules immutable as a deterrent to users accidently
      installing a kernel (which would only waste space).
	
    \item
      Configure vncserver to start a desktop for the main user (off
      by default).

    \item
      Make the location of the vserver a symlink to a directory
      whose name matches the context id. (security measure)

    \item
      Set permissions to /vservers to 0000. (Always, just in case).
      
    \end{itemize}

\end{itemize}
