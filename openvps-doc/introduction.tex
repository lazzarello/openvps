\chapter{Introduction\label{introduction}}

\section{Virtual Private Servers\label{intr-vservers}}

A traditional hosting provider provides disk space accessible via its
web server to store content in the form of static HTML pages. The
pages are typically uploaded via FTP. More advanced hosting providers
offer additional services such as ability to run CGI scripts as well
as mail and DNS hosting. The software running such services is
administered by the provider with only limited (and usually
proprietary) interface offered to the user.

With the recent advances in WWW content publishing technologies, the
popularity of LAMP (Linux, Apache, MySQL and Perl/Python/PHP) hosting
there is a clear demand for more control by the users. Users want to
be able to compile and install the software they are running and be
able to configure it directly. Traditionally, this flexibility was
only available on dedicated servers, where the user is given
administrative (root) access and a free-range on the server, however,
the cost of a dedicated server is often prohibitive compared to
budgets allocated for projects. (While the cost of dedicated servers
has declined to very low levels, the quality and reliability of such a
server and fiscal soundness of such a provider should probably be
questioned.)

For many years now there existed an alternative solution to this
problem, known as \dfn{Virtual Private Servers} (VPS), also often
referred to as \dfn{Virtual Dedicated Servers} (VDS).

A Virtual Private Server is a complete environment which behaves in
nearly every respect as a dedicated server. A user may log in as root,
start and stop daemons, add or remove users, compile software, etc,
all without even being aware that the server is actually
\emph{virtual}.

There are several technologies currently availale to provide
virtualization. Some are based on a software simulation of a physical
machine (such as QEMU and Bochs), others on running a kernel in
user-mode (UML), while others on in-kernel contextualization of
processes (Linux VServer). 

\begin{seealso}
  \seeurl{http://www.kernelthread.com/publications/virtualization/}{``An Introduction to Virtualization'' by Amit Singh}
  \seeurl{http://www.linux-vserver.org/}{Linux VServer Project}
  \seeurl{http://user-mode-linux.sourceforge.net/}{User Mode Linux}
  \seeurl{http://fabrice.bellard.free.fr/qemu/}{QEMU}
  \seeurl{http://bochs.sourceforge.net/}{Bochs}
\end{seealso}

\section{Target Audience\label{intr-target}}

Primarily, the target user-base for the OpenVPS project is web-hosting
providers (commercial or otherwise) and users (webmasters). It is our
hope that as the project matures, the demand for this software will
grow from both providers and their clients as they begin to appreciate
the advantages of shared knowledge and openness in software.

\section{OpenVPS\label{intr-openvps}}

While most of the aforementioned virtualization technologies solve the
most difficult issue of the actual virtualization, taking that one
step further to a complete, flexible and scalable hosting environment
is still quite an undertaking. Additional requirements such as
collection of bandwidth utilization, backups, customization, ability
to manage virtual servers across multiple physical servers,
maintaining software and monitoring security etc, are typically left
as an ``exersise for the reader'' by most virtualization packages
(while some serve to promote commercial software aimed at web
hosters).

\dfn{OpenVPS} project's goal is to provide a complete open source
solution for hosting virtual servers. At the time of this writing, it
is believed to be the only free and open source project to address the
needs of hosting providers.

At the initial stage of the project, OpenVPS is specific to the Linux
operating system, using the Linux VServer project as the underlying
virtualization technology. 

The Linux VServer separates processes by assigning a \dfn{context id}
which allows creation of virtual Linux environments with practically
no overhead. The low overhead and therefore stellar performance
caracteristics were the prime motivator behind selecting it as the
base technology for OpenVPS.

The OS platform is the RedHat Fedora Core Linux distribution. Because
of the fine-grain detailed approach of OpenVPS, it does not lend
itself to distribution-independence.

At this stage of the project, OpenVPS provides the following:

\begin{itemize}

\item
  Configuration of the physical server to make it suitable for
  hosting. (Disabling of services, security tightening, etc.)

\item
  Building of a \dfn{Reference Server}. The reference server serves
  as the starting template for all virtual servers. The reference
  server consists of a large selection of Fedora packages chosen
  based on applicability to a hosting environment. This list can be
  augmented via the OpenVPS configuration file.

\item
  \dfn{Cloning} of the Reference Server. This is a process by which
  the virtual server is created. The actual cloning means either
  copying, hard-linking or touching a file depending on its VServer
  flags and the OpenVPS clone rules. (The iunlink flag is part of
  the VServer \dfn{unification} feature).

\item
  \dfn{Customization} of the newly-cloned VPS. This is where the final
  touches are applied such as the IP, the username, passwords, SSL
  certs and SSH keys, etc..

\item
  Accounting of VPS bandwidth using iptables and an RRD. 

\item
  Providing a WWW interface to allow viewing of real-time charts of
  bandwidth utilization data.

\item
  Nightly backups to a local partition or an outside server.

\end{itemize}

\section{Prerequisites\label{intr-prereq}}

The main prerequisite is a server that will be dedicated to
hosting. The OpenVPS configuration script will insist on certain
server configuration, esp. when it comes to sshd, httpd, named and the
iptables, which makes it not very usable for other purposes.

Since each VPS uses at least one IP address, you need access to a
range of IP addresses. 







